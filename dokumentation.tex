\documentclass[11pt,a4paper]{article}

% Pakete
\usepackage[utf8]{inputenc}
\usepackage[T1]{fontenc}
\usepackage[ngerman]{babel}
\usepackage{geometry}
\usepackage{graphicx}
\usepackage{xcolor}
\usepackage{listings}
\usepackage{booktabs}
\usepackage{hyperref}
\usepackage{fancyhdr}
\usepackage{titlesec}
\usepackage{tcolorbox}
\usepackage{enumitem}

% Seitenränder
\geometry{left=2.5cm, right=2.5cm, top=2.5cm, bottom=2.5cm}

% Farben
\definecolor{codeblue}{RGB}{0,102,204}
\definecolor{codegray}{RGB}{128,128,128}
\definecolor{codegreen}{RGB}{0,128,0}
\definecolor{backcolour}{RGB}{245,245,245}
\definecolor{accentcolor}{RGB}{0,82,147}

% Code-Stil
\lstdefinestyle{mystyle}{
    backgroundcolor=\color{backcolour},
    commentstyle=\color{codegreen},
    keywordstyle=\color{codeblue}\bfseries,
    numberstyle=\tiny\color{codegray},
    stringstyle=\color{codegreen},
    basicstyle=\ttfamily\small,
    breakatwhitespace=false,
    breaklines=true,
    captionpos=b,
    keepspaces=true,
    numbers=left,
    numbersep=5pt,
    showspaces=false,
    showstringspaces=false,
    showtabs=false,
    tabsize=2,
    frame=single,
    rulecolor=\color{codegray}
}
\lstset{style=mystyle}

% Kopf- und Fußzeile
\pagestyle{fancy}
\fancyhf{}
\fancyhead[L]{\textcolor{accentcolor}{PDF zu CSV -- Technische Dokumentation}}
\fancyhead[R]{\textcolor{accentcolor}{\thepage}}
\fancyfoot[C]{\small BeeZubi Lernwelt GmbH}
\renewcommand{\headrulewidth}{0.4pt}
\renewcommand{\footrulewidth}{0.4pt}

% Titelformatierung
\titleformat{\section}{\Large\bfseries\color{accentcolor}}{\thesection}{1em}{}
\titleformat{\subsection}{\large\bfseries\color{accentcolor}}{\thesubsection}{1em}{}

% Hyperlink-Einstellungen
\hypersetup{
    colorlinks=true,
    linkcolor=accentcolor,
    filecolor=accentcolor,
    urlcolor=accentcolor,
}

% Box-Stil für Hinweise
\tcbuselibrary{skins,breakable}
\newtcolorbox{infobox}[1][]{
    colback=blue!5!white,
    colframe=accentcolor,
    fonttitle=\bfseries,
    title=#1,
    breakable
}

\newtcolorbox{warningbox}[1][]{
    colback=orange!5!white,
    colframe=orange!75!black,
    fonttitle=\bfseries,
    title=#1,
    breakable
}

% Dokumentbeginn
\begin{document}

% Titelseite
\begin{titlepage}
    \centering
    \vspace*{2cm}

    {\Huge\bfseries\textcolor{accentcolor}{PDF zu CSV}}\\[0.5cm]
    {\Large Automatisierung der IHK-Fragengenerierung}\\[2cm]

    {\large\textbf{Technische Dokumentation}}\\[0.5cm]
    {\large Version 1.0}\\[3cm]

    \begin{tabular}{ll}
        \textbf{Technologie:} & OpenAI GPT-4.1 Vision API \\[0.3cm]
        \textbf{Sprache:} & Python 3.12+ \\[0.3cm]
        \textbf{Ausgabeformat:} & CSV (UTF-8 mit BOM) \\[0.3cm]
    \end{tabular}

    \vfill

    {\large BeeZubi Lernwelt GmbH}\\[0.3cm]
    {\small Januar 2026}
\end{titlepage}

% Inhaltsverzeichnis
\tableofcontents
\newpage

% ============================================
\section{Einführung}
% ============================================

Das Programm \texttt{pdf\_to\_csv.py} automatisiert die Erstellung von IHK-Prüfungsfragen aus PDF-Lehrbüchern. Es verwendet die OpenAI Vision API, um Buchseiten zu analysieren und daraus Single-Choice-Fragen im CSV-Format zu generieren.

\subsection{Anwendungsbereich}

\begin{itemize}[leftmargin=*]
    \item Kaufleute im Einzelhandel
    \item Kaufleute für Büromanagement
    \item Weitere IHK-Ausbildungsberufe (durch Anpassung des Megaprompts)
\end{itemize}

\subsection{Hauptfunktionen}

\begin{enumerate}[leftmargin=*]
    \item PDF-Seiten in Bilder konvertieren
    \item Bilder an OpenAI Vision API senden
    \item 12--15 Prüfungsfragen pro Doppelseite generieren
    \item Ergebnisse als CSV speichern (Excel-kompatibel)
\end{enumerate}

% ============================================
\section{Systemanforderungen}
% ============================================

\subsection{Software}

\begin{table}[h]
\centering
\begin{tabular}{@{}lll@{}}
\toprule
\textbf{Komponente} & \textbf{Version} & \textbf{Zweck} \\
\midrule
Python & 3.12+ & Laufzeitumgebung \\
PyMuPDF (fitz) & aktuell & PDF-Verarbeitung \\
OpenAI & aktuell & API-Kommunikation \\
python-docx & aktuell & DOCX-Dateien lesen \\
python-dotenv & aktuell & Umgebungsvariablen \\
\bottomrule
\end{tabular}
\caption{Erforderliche Software-Komponenten}
\end{table}

\subsection{API-Zugang}

\begin{itemize}[leftmargin=*]
    \item OpenAI API-Schlüssel erforderlich
    \item Schlüssel in \texttt{.env}-Datei speichern
\end{itemize}

% ============================================
\section{Installation}
% ============================================

\subsection{Virtuelle Umgebung erstellen}

\begin{lstlisting}[language=bash]
# Virtuelle Umgebung erstellen
python -m venv venv

# Aktivieren (Windows)
venv\Scripts\activate

# Aktivieren (Linux/Mac)
source venv/bin/activate
\end{lstlisting}

\subsection{Abhängigkeiten installieren}

\begin{lstlisting}[language=bash]
pip install -r requirements.txt
\end{lstlisting}

\subsection{API-Schlüssel konfigurieren}

Erstellen Sie eine Datei \texttt{.env} im Projektverzeichnis:

\begin{lstlisting}
OPENAI_API_KEY=sk-ihr-api-schluessel-hier
\end{lstlisting}

\begin{warningbox}[Sicherheitshinweis]
Der API-Schlüssel sollte niemals in Versionskontrollsysteme eingecheckt werden. Die Datei \texttt{.env} ist in \texttt{.gitignore} aufzunehmen.
\end{warningbox}

% ============================================
\section{Programmarchitektur}
% ============================================

\subsection{Ablaufdiagramm}

\begin{center}
\begin{tcolorbox}[width=0.9\textwidth, colback=white, colframe=accentcolor]
\centering
\textbf{Verarbeitungsablauf}\\[0.5cm]
\begin{tabular}{c}
PDF-Datei laden \\
$\downarrow$ \\
Megaprompt aus DOCX laden \\
$\downarrow$ \\
\textit{Für jede Seite:} \\
$\downarrow$ \\
Seite $\rightarrow$ PNG-Bild (150 DPI) \\
$\downarrow$ \\
Bild $\rightarrow$ Base64 kodieren \\
$\downarrow$ \\
Megaprompt + Bild $\rightarrow$ OpenAI API \\
$\downarrow$ \\
Antwort validieren \& bereinigen \\
$\downarrow$ \\
CSV-Zeilen speichern \\
\end{tabular}
\end{tcolorbox}
\end{center}

\subsection{Kernkomponenten}

\begin{table}[h]
\centering
\begin{tabular}{@{}lp{8cm}@{}}
\toprule
\textbf{Funktion} & \textbf{Beschreibung} \\
\midrule
\texttt{lade\_megaprompt()} & Lädt den Prompt aus einer DOCX-Datei \\
\texttt{pdf\_seite\_zu\_base64()} & Konvertiert PDF-Seite in Base64-Bild \\
\texttt{rufe\_openai\_vision()} & Sendet Anfrage an OpenAI API \\
\texttt{bereinige\_und\_validiere\_csv()} & Prüft und korrigiert CSV-Format \\
\texttt{verarbeite\_seite()} & Orchestriert die Seitenverarbeitung \\
\bottomrule
\end{tabular}
\caption{Hauptfunktionen des Programms}
\end{table}

\subsection{Fehlerbehandlung}

Das Programm implementiert eine dreistufige Strategie:

\begin{enumerate}[leftmargin=*]
    \item \textbf{Megaprompt}: Vollständiger Prompt aus DOCX-Datei
    \item \textbf{Standard}: Vereinfachter Fallback-Prompt
    \item \textbf{Notfall}: Minimaler Prompt für problematische Seiten
\end{enumerate}

Bei Verbindungsfehlern werden bis zu 5 Wiederholungen mit exponentiellem Backoff durchgeführt.

% ============================================
\section{Verwendung}
% ============================================

\subsection{Grundlegende Syntax}

\begin{lstlisting}[language=bash]
python pdf_to_csv.py --pdf <PDF-DATEI> --prompt <MEGAPROMPT.docx> [OPTIONEN]
\end{lstlisting}

\subsection{Parameter}

\begin{table}[h]
\centering
\begin{tabular}{@{}llp{6cm}@{}}
\toprule
\textbf{Parameter} & \textbf{Standard} & \textbf{Beschreibung} \\
\midrule
\texttt{--pdf} & -- & Pfad zur PDF-Datei (erforderlich) \\
\texttt{--prompt} & -- & Pfad zur Megaprompt-DOCX (erforderlich) \\
\texttt{--start} & 1 & Erste zu verarbeitende Seite \\
\texttt{--end} & letzte & Letzte zu verarbeitende Seite \\
\texttt{--output} & auto & Name der Ausgabe-CSV \\
\texttt{--model} & gpt-4.1 & OpenAI-Modell \\
\texttt{--parallel} & 1 & Anzahl paralleler Anfragen \\
\texttt{--test} & -- & Nur 2 Seiten verarbeiten \\
\texttt{--book-start} & auto & Erste Buchseitennummer \\
\bottomrule
\end{tabular}
\caption{Kommandozeilenparameter}
\end{table}

\subsection{Verfügbare Modelle}

\begin{table}[h]
\centering
\begin{tabular}{@{}llll@{}}
\toprule
\textbf{Modell} & \textbf{Qualität} & \textbf{Geschwindigkeit} & \textbf{Kosten} \\
\midrule
gpt-4.1 & Beste & Mittel & \$2.00/1M Token \\
gpt-4.1-mini & Sehr gut & Schnell & \$0.40/1M Token \\
gpt-4.1-nano & Gut & Sehr schnell & \$0.10/1M Token \\
gpt-4o & Sehr gut & Mittel & \$2.50/1M Token \\
gpt-4o-mini & Gut & Schnell & \$0.15/1M Token \\
\bottomrule
\end{tabular}
\caption{Verfügbare OpenAI-Modelle}
\end{table}

\subsection{Beispiele}

\begin{infobox}[Standardverwendung]
\begin{lstlisting}[language=bash]
python pdf_to_csv.py --pdf Lehrbuch.pdf --prompt "Megaprompt EH.docx"
\end{lstlisting}
\end{infobox}

\begin{infobox}[Seitenbereich mit paralleler Verarbeitung]
\begin{lstlisting}[language=bash]
python pdf_to_csv.py --pdf Lehrbuch.pdf --prompt "Megaprompt EH.docx" \
    --start 50 --end 100 --parallel 3
\end{lstlisting}
\end{infobox}

\begin{infobox}[Testmodus (nur 2 Seiten)]
\begin{lstlisting}[language=bash]
python pdf_to_csv.py --pdf Lehrbuch.pdf --prompt "Megaprompt EH.docx" --test
\end{lstlisting}
\end{infobox}

% ============================================
\section{Ausgabeformat}
% ============================================

\subsection{CSV-Struktur}

Die generierte CSV-Datei verwendet:
\begin{itemize}[leftmargin=*]
    \item \textbf{Kodierung}: UTF-8 mit BOM (Excel-kompatibel)
    \item \textbf{Trennzeichen}: Semikolon (\texttt{;})
    \item \textbf{Kopfzeile}: Ja (erste Zeile)
\end{itemize}

\subsection{Spalten}

\begin{table}[h]
\centering
\begin{tabular}{@{}clp{6cm}@{}}
\toprule
\textbf{Nr.} & \textbf{Spalte} & \textbf{Beschreibung} \\
\midrule
1 & Frage & Die Prüfungsfrage \\
2 & A & Antwortmöglichkeit A \\
3 & B & Antwortmöglichkeit B \\
4 & C & Antwortmöglichkeit C \\
5 & D & Antwortmöglichkeit D \\
6 & Richtig & Richtige Antwort (A/B/C/D) \\
7 & Richtig\_Text & Ausgeschriebene richtige Antwort \\
8 & Thema & Unterkapitel/Stichwort \\
9 & Quelle & Buchseiten (z.B. "`Buch S. 17--18"') \\
10 & Status & Prüfstatus ("`ok"' oder "`prüfen"') \\
11 & Kommentar & Optionaler Kommentar \\
12 & Vollansicht & Kurz-Lesefassung \\
\bottomrule
\end{tabular}
\caption{CSV-Spaltenstruktur}
\end{table}

% ============================================
\section{Megaprompt-Konfiguration}
% ============================================

Der Megaprompt ist eine DOCX-Datei, die alle Anweisungen für die Fragengenerierung enthält.

\subsection{Wichtige Elemente}

\begin{enumerate}[leftmargin=*]
    \item \textbf{Rolle}: Definition der KI als Prüfungsexpertin
    \item \textbf{Fragentypen}: W-Frage, Operator-Aufgabe, Fallvignette (im Wechsel)
    \item \textbf{Betriebe \& Personen}: Feste Namen für Fallbeispiele
    \item \textbf{Formatregeln}: CSV-Struktur, keine Semikolons im Text
    \item \textbf{Qualitätskriterien}: IHK-Niveau, Praxisbezug
\end{enumerate}

\subsection{Anpassung für andere Berufe}

Um das Programm für andere Ausbildungsberufe zu verwenden:

\begin{enumerate}[leftmargin=*]
    \item Kopieren Sie den vorhandenen Megaprompt
    \item Passen Sie Betriebe und Personennamen an
    \item Ändern Sie fachspezifische Begriffe
    \item Speichern Sie als neue DOCX-Datei
\end{enumerate}

% ============================================
\section{Fehlerprotokollierung}
% ============================================

Das Programm erstellt automatisch Fehler-Logs:

\begin{itemize}[leftmargin=*]
    \item \texttt{<pdf-name>\_errors.json} -- Strukturiertes JSON-Log
    \item \texttt{<pdf-name>\_errors.txt} -- Lesbares Text-Log
\end{itemize}

\subsection{Fehlertypen}

\begin{table}[h]
\centering
\begin{tabular}{@{}lp{7cm}@{}}
\toprule
\textbf{Code} & \textbf{Beschreibung} \\
\midrule
CONTENT\_POLICY & API hat Inhalt abgelehnt \\
KEIN\_TEXT & Seite enthält nicht genug Text \\
NUR\_BILD & Seite enthält nur Bilder/Diagramme \\
UNGUELTIG\_CSV & Antwort nicht im CSV-Format \\
VERBINDUNG & Netzwerkfehler \\
RATE\_LIMIT & API-Anfragelimit erreicht \\
ZEITÜBERSCHREITUNG & Timeout überschritten \\
\bottomrule
\end{tabular}
\caption{Mögliche Fehlertypen}
\end{table}

% ============================================
\section{Best Practices}
% ============================================

\begin{infobox}[Empfehlungen für optimale Ergebnisse]
\begin{enumerate}[leftmargin=*]
    \item \textbf{OCR-Qualität}: Verwenden Sie gut gescannte PDFs mit OCR
    \item \textbf{Parallele Verarbeitung}: Maximal 3 parallele Anfragen
    \item \textbf{Testlauf}: Immer zuerst \texttt{--test} verwenden
    \item \textbf{Seitenbereich}: Bei großen PDFs in Abschnitten verarbeiten
    \item \textbf{Validierung}: Generierte Fragen manuell prüfen
\end{enumerate}
\end{infobox}

\begin{warningbox}[Wichtige Hinweise]
\begin{itemize}[leftmargin=*]
    \item API-Kosten beachten (ca. \$0.02 pro Seite bei gpt-4.1)
    \item Keine Semikolons in Fragen/Antworten verwenden
    \item Bei Verbindungsproblemen \texttt{--parallel 1} nutzen
\end{itemize}
\end{warningbox}

% ============================================
\section{Technische Details}
% ============================================

\subsection{Bildkonvertierung}

\begin{itemize}[leftmargin=*]
    \item \textbf{Auflösung}: 150 DPI (konfigurierbar)
    \item \textbf{Format}: PNG
    \item \textbf{Kodierung}: Base64 für API-Übertragung
\end{itemize}

\subsection{API-Parameter}

\begin{lstlisting}[language=Python]
client.chat.completions.create(
    model="gpt-4.1",
    messages=[...],
    max_tokens=4096,
    temperature=0.7,
    timeout=120
)
\end{lstlisting}

\subsection{Retry-Logik}

\begin{itemize}[leftmargin=*]
    \item Maximale Versuche: 5
    \item Basis-Wartezeit: 2 Sekunden
    \item Maximale Wartezeit: 60 Sekunden
    \item Exponentieller Backoff mit Jitter
\end{itemize}

% ============================================
\section{Anhang}
% ============================================

\subsection{Projektstruktur}

\begin{lstlisting}
PDF_in_csv_automation_mit_APIKeys/
    pdf_to_csv.py          # Hauptprogramm
    requirements.txt        # Abhangigkeiten
    .env                   # API-Schlussel (nicht versioniert)
    .env.example           # Vorlage fur .env
    Megaprompt *.docx      # Prompt-Dateien
    *.pdf                  # Eingabe-PDFs
    *_questions.csv        # Generierte Fragen
    *_errors.json          # Fehler-Logs
\end{lstlisting}

\subsection{Kontakt}

Bei Fragen oder Problemen wenden Sie sich an:\\
\textbf{BeeZubi Lernwelt GmbH}

\vfill

\begin{center}
\textcolor{codegray}{\small Dokumentation erstellt: Januar 2026}
\end{center}

\end{document}
